\documentclass[a4paper,12pt, english]{article}
\usepackage[T1]{fontenc}
\usepackage[utf8]{inputenc}
\usepackage{graphicx}
\usepackage{babel}
\usepackage{amsmath}
\usepackage{ulem}
\usepackage{a4wide}
\usepackage{graphicx}
\usepackage{listings}
\usepackage{tabularx}
\usepackage{tabulary}

\begin{document}

\begin{titlepage}
\begin{center}
\textsc{\Large Computational Physics, Project 4}\\[0.5cm]
\textsc{Vilde Eide Skingen, John-Anders Stende and Kari Eriksen}\\[0.5cm]

\end{center}
\end{titlepage}

\begin{abstract}

simulate how the signals between neurons in the brain 

\end{abstract}

\section*{Diffusion of neruotransmitters in the synaptic cleft}

\subsection*{Method}

In this project we want as mentioned simulate the concentration of a particular neurotransmitter in the synaptic cleft. More specific between the presynaptic membrane and the postsynaptic membrane. \
In order to compare our simulation, using the three different methods Backward Euler, Forward Euler and the Crank-Nicolson scheme, we need something we know to be an exact solution or a closed form solution. One of our task in this project is to find this solution. Which in this case is on closed form. 
\\
We start with the diffusion equation in 1 dimension
\\
\begin{equation}
\frac{\partial u}{\partial t} = D\frac{\partial^2 u}{\partial x^2}.
\label{eq:diffusion_1d}
\end{equation}	
\\

We are given the solution of the steady-state 
\\
$$u_s(x) = 1 - x,$$
\\
which obeys the boundary conditions for Eq.~(\ref{eq:diffusion_1d}). We therefore do not need to worry about these previous conditions on the solution $u.$ We make a new function
\\
$$v(x,t) = u(x,t) - u_s(x)$$ 
\\
with boundary conditions $v(0) = v(d) = 0$ and initial condition $v(x,0) = u(x,0) - u_s(x).$ And this function $v(x,t)$ is easier to solve.
By first solving $v(x,t)$ it is easy to find $u(x,t).$ $u(x,t)$ is simply $v(x,t) + u_s(x)$ where $u_s(x)$ is time-independent, and therefore can just be added after we we know $v.$
To find $v$ we first use separation of variables. 
\\
$$v(x,t) = X(x)T(t)$$
\\
$$\frac{\partial v(x,t)}{\partial t} = X(x) \frac{\partial T(t)}{\partial t}$$
\\
$$\frac{\partial^2 v(x,t)}{\partial x^2} = T(t) \frac{\partial^2 X(x)}{\partial x^2}$$
\\
$$X(x)\dot{T}(t) = T(t)\ddot{X}(x)$$
\\
$$\frac{\dot{T}(t)}{T(t)} = \frac{\ddot{X}(x)}{X(x)}$$
\\
For this to be true for all $t$ and $x,$ both the rhs and the lhs must be constant.
\\
$$\frac{\dot{T}(t)}{T(t)} = k \Rightarrow \dot{T}(t) - kT(t) = 0$$
\\
$$\frac{\ddot{X}(x)}{X(x)} = k \Rightarrow \ddot{X}(x) - kX(x) = 0$$
\\
First we solve:
\\
$$\ddot{X}(x) - kX(x) = 0.$$
\\
Using the abc-method we find that $X = \pm \sqrt{k}.$
We use $k = \mu^2$ and get $\sqrt{k} = \mu.$
\\
For $\mu^2 \gg 0$ the general solution becomes
\\
$$X(x) = Ae^{\mu x} + Be^{-\mu x}.$$
\\
Using boundary conditions $v(0) = v(d) = 0$ 
\\
$$X(0) = Ae^0 + Be^0 = 0$$
$$\Rightarrow A + B = 0$$ and
$$X(d) = Ae^{\mu d} + Be^{-\mu d} = 0$$
$$\Rightarrow A = B = 0.$$
\\
This solution is not of interest. We instead look at $k = 0.$ The general solution in this case is
\\
$$X(x) = ax + b.$$
\\
With boundary conditions we get
\\
$$X(0) = b = 0$$
$$X(d) = ad = 0 \Rightarrow a = 0.$$
\\
The only case of interest is then $k = -\mu^2$, when $-\mu^2 \ll 0$. We get the general solution
\\
$$X(x) = A\cos(\mu x) + B\sin(\mu x).$$
\\
Boundary conditions give
\\
$$X(0) = A\cos(0) + B\sin(0) = A = 0$$
$$X(d) = B\sin(\mu d) \Rightarrow \sin(\mu d) = 0.$$
\\
We know that $\sin$ is zero when $x  = 0, \pi, 2\pi,...$, we therefore need $\mu d = n \pi$. We then get the solution for X,
\\
$$X_n(x) = \sum_{n=1}^{\infty} A_n \sin\left(n\frac{\pi}{d} x\right)$$ 
\\
which of course is zero in the end points $x = 0$ and $x = d = 1$.
\\
\\
Now we solve:
\\
$$\dot{T}(t) - kT(t) = 0.$$
\\
We solve for $T$, and get $T = k$. The gereral solution then becomes
\\
$$T(t) = Be^{-k2 t} = Be^{-\mu^2 t}.$$
\\
Since $\mu d = n \pi$,
\\
$$T(t) = Be^{-\left( n \frac{\pi}{d}\right)^2 t}.$$
\\
Now we can write out the solution of $v$. 
\\
$$v(x,t) = \sum_{n=1}^{\infty}v_n(x,t) =
\sum_{n=1}^{\infty} B_n e^{-\left( n \frac{\pi}{d}\right)^2 t} A_n \sin\left(n\frac{\pi}{d} x\right)$$
\\
$$A_nB_n = C_n$$
\\
We use the initial condition 
\\
$$v(x,0) = \sum_{n=1}^{\infty} C_n \sin\left(n\frac{\pi}{d} x\right) = f(x) = x -1.$$
\\
We recognize $C_n$ as the inverse fourier coefficient to the function $f(x)$.
Thus we get
\\
$$C_n = 2\int_0^1 \mathrm(x - 1)\sin (n\pi x)\,\mathrm{d}x.$$
\\
$$\int_0^1 \mathrm x \sin n \pi x\,\mathrm{d}x = \frac{1}{n\pi}$$
\\
$$-\int_0^1 \mathrm \sin \pi x\,\mathrm{d}x = \frac{2}{n \pi}$$
\\
$$C_n = 2 \left(\frac{1}{n\pi} - \frac{2}{n\pi} \right) = -\frac{2}{n\pi}$$
\\
Here we have set $d=1$. Now the final solution becomes
\\
$$v(x,t) = \sum_{n=1}^{\infty} C_n e^{-\left( n \pi\right)^2 t} \sin(n \pi x),$$
$$v(x,t) = -2 \sum_{n=1}^{\infty} \frac{1}{n \pi} e^{-\left( n \pi\right)^2 t} \sin(n \pi x).$$
\\
Now it is easy to find  $u$, when it is simply
\\
$$u(x,t) = v(x,t) + u_s(x),$$
\\
$$u(x,t) = (1 - x) -2 \sum_{n=1}^{\infty} \frac{1}{n \pi} e^{-\left( n \pi\right)^2 t} \sin(n \pi x).$$
\\
Now we have the closed form solution to our problem, but we also want to solve it numerically by using three different methods. The implicit Backward-Euler scheme, the explicit Forward-Euler scheme and the implicit Crank-Nicolson scheme. 
\\
The algorithms for these three methods goes as follows:

\subsubsection*{Backward Euler}
$$u_t \approx \frac{u(x_i,t_j) - u(x_i,t_j - \Delta t)}{\Delta t}$$
\\
$$u_{xx} \approx \frac{u(x_i + \Delta x,t_j) - 2u(x_i,t_j) + u(x_i - \Delta x,t_j)}{\Delta x^2}$$

\subsubsection*{Forward Euler}
$$u_t \approx \frac{u(x_i,t_j + \Delta t) - u(x_i,t_j)}{\Delta t}$$
\\
$$u_{xx} \approx \frac{u(x_i + \Delta x,t_j) - 2u(x_i,t_j) + u(x_i - \Delta x,t_j)}{\Delta x^2}$$

\subsubsection*{Crank-Nicolson}
$$u_t \approx \frac{u(x_i,t_j + \Delta t) - u(x_i,t_j)}{\Delta t}$$
\\
$$u_{xx} \approx \frac{1}{2} \left(\frac{u(x_i + \Delta x,t_j) - 2u(x_i,t_j) + u(x_i - \Delta x,t_j)}{\Delta x^2} +$$
 
$$\frac{u(x_i + \Delta x,t_j + \Delta t) - 2u(x_i,t_j+ \Delta t) + u(x_i - \Delta x,t_j + \Delta t)}{\Delta x^2}\right)$$
















\end{document}